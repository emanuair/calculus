\documentclass{ximera}

\input{../preamble.tex}

\outcome{Use the first derivative to determine whether a function is increasing or decreasing.}
\outcome{Identify the relationships between the function and its first and second derivatives.}
\outcome{Sketch a graph of the second derivative, given the original function.}
\outcome{Sketch a graph of the original function, given the graph of its first and second derivatives.}
\outcome{State the relationship between concavity and the second derivative.}

\title[Dig-In:]{Concavity}

\begin{document}
\begin{abstract}
  Here we examine what the second derivative tells us about the
  geometry of functions.
\end{abstract}
\maketitle

The graphs of two functions, $f$ and $g$, both increasing on the given interval, are given below.
\begin{image}
    \includegraphics{6.png}
\end{image}


\begin{definition} Let $f$ be a  function differentiable on an open interval $I$.
  \begin{itemize}
    \item We say that the graph of $f$ is \dfn{concave up} on $I$
      if $f'$, the derivative of $f$, is \textit{increasing} on
      $I$.
    \item We say that the graph of $f$ is \dfn{concave down} on
      $I$ if $f'$, the derivative of $f$, is \textit{decreasing} on
      $I$.
  \end{itemize}
\end{definition}

We know that the sign of the derivative tells us whether a function is
increasing or decreasing at some point. Likewise, the sign of the
second derivative $f''(x)$ tells us whether $f'(x)$ is increasing or
decreasing at $x$. 
If we are trying to understand the shape of the graph of a function,
knowing where it is concave up and concave down helps us to get a more
accurate picture. This is  summarized in a single theorem.


\begin{theorem}[Test for Concavity]\index{concavity test} Let $I$ be an open interval.
\begin{itemize}
\item If $f''(x)>0$ for all $x$ in $I$, then the graph of $f$ is  concave up on $I$.
\item If $f''(x)<0$ for all $x$ in $I$, then the graph of $f$ is  concave down on $I$.
\end{itemize}
\end{theorem}
We summarize the consequences of this theorem  in the table below:


\begin{image}
  \includegraphics{7.png}
\end{image}


\begin{example}
  Let $f$ be a continuous function and suppose that:
  \begin{itemize}
  \item $f'(x) > 0$ for $-1< x<1$.
  \item $f'(x) < 0$ for $-2< x<-1$ and $1<x<2$.
  \item $f''(x) > 0$ for $-2<x<0$ and $1<x< 2$.
  \item $f''(x) < 0$ for $0<x< 1$.  
  \end{itemize}
  Sketch a possible graph of $f$.
  \begin{explanation}
    Start by marking points in the domain where the derivative changes sign and indicate
    intervals where $f$ is increasing and intervals $f$ is
    decreasing. The function $f$ has a negative derivative from $-2$
    to $x=\answer[given]{-1}$. This means that $f$ is
    \wordChoice{\choice{increasing}\choice[correct]{decreasing}} on
    this interval. The function $f$ has a positive derivative from
    $x=\answer[given]{-1}$ to $x=\answer[given]{1}$. This means that
    $f$ is
    \wordChoice{\choice[correct]{increasing}\choice{decreasing}} on
    this interval. Finally, The function $f$ has a negative derivative
    from $x=\answer[given]{1}$ to $2$. This means that $f$ is
    \wordChoice{\choice{increasing}\choice[correct]{decreasing}} on
    this interval.
  \begin{image}
    \includegraphics{8.png}
  \end{image}
  Now we should sketch the concavity: \wordChoice{\choice[correct]{concave up}\choice{concave down}} when the second
  derivative is positive, \wordChoice{\choice{concave up}\choice[correct]{concave down}} when the second derivative is
  negative.
    \begin{image}
      \includegraphics{9.png}
    \end{image}
    Finally, we can sketch our curve:
        \begin{image}
      \includegraphics{10.png}
  \end{image}
  \end{explanation}
\end{example}

\end{document}
