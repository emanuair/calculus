\documentclass{ximera}

\input{../preamble.tex}


\outcome{Find derivatives of inverse functions in general.}
\outcome{Understand how the derivative of an inverse function relates to the original derivative.}
\outcome{Explain the formula for the derivative of the natural log function.}
\outcome{Calculate derivatives of logs in any base.}

\title[Dig-In:]{Derivatives of inverse exponential functions}

\begin{document}
\begin{abstract}
  We derive the derivatives of inverse exponential functions using
  implicit differentiation.
\end{abstract}
\maketitle


Geometrically, there is a close relationship between the plots of
$e^x$ and $\ln(x)$, they are reflections of each other over the line
$y=x$:
\begin{image}
  \includegraphics{5.png}
%% \caption{A plot of $e^x$ and $\ln(x)$. Since they are inverse
%%   functions, they are reflections of each other across the line $y=x$.}
%% \label{plot:e^x lnx}
\end{image}
One may suspect that we can use the fact that $\ddx e^x = e^x$, to
deduce the derivative of $\ln(x)$.  We will use implicit
differentiation to exploit this relationship computationally.

\begin{theorem}[The Derivative of the Natural Logrithm]\index{derivative!of the natural logarithm}
\[
\ddx \ln(x) = \frac{1}{x}.
\]
\begin{explanation}
 %% To start, note that the Inverse Function Theorem assures us that this
  %% derivative actually exists.
  Recall
\[
\ln(x) = y \qquad\text{exactly when}\qquad e^y = x\qquad\text{and}\qquad x>\answer[given]{0}.
\]
Hence
\begin{align*}
e^y &= x\\
\ddx e^y &= \ddx x &\text{Differentiate both sides.}\\
e^y \dd[y]{x} &= 1 &\text{Implicit differentiation.}\\
\dd[y]{x} &= \frac{1}{e^y} = \answer[given]{\frac{1}{x}}&\text{Solve for $\dd[y]{x}$.}
\end{align*}
Since $y=\ln(x)$, $\ddx \ln(x) = \answer[given]{\frac{1}{x}}$.
\end{explanation}
\end{theorem}

\begin{question}
  Compute:
  \[
  \ddx \left(-\ln(\cos(x))\right)
  \begin{prompt}
    = \answer{\tan(x)}
  \end{prompt}
  \]
\end{question}



From the derivative of the natural logarithm, we can deduce another fact:

\begin{theorem}[The derivative of any logarithm]
  Let $b$ be a positive real number. Then
  \[
  \ddx \log_b(x) = \frac{1}{x\ln(b)}.
  \]
  \begin{explanation}
    Here we need to remember that
    \[
    \log_b(x) = \frac{\ln(x)}{\answer[given]{\ln(b)}}.
    \]
    So we may write
    \begin{align*}
      \ddx \log_b(x) &= \ddx \frac{\ln(x)}{\answer[given]{\ln(b)}}\\
      &= \answer[given]{\frac{1}{x\ln(b)}}.
    \end{align*}
  \end{explanation}
\end{theorem}

\begin{question}
  Compute:
  \[
  \ddx \log_7(x)
  \begin{prompt}
    = \answer{1/(x \ln(7))}
  \end{prompt}
  \]
\end{question}


We can also compute the derivative of an arbitrary exponential
function.

\begin{theorem}[The derivative of an exponential function]
  \[
  \ddx a^x = a^x\cdot \ln(a).
  \]
  \begin{explanation}
    Here we need to be slightly sneaky. Note
    \[
    a^x = e^{\ln(a^x)} = e^{x\ln(a)}.
    \]
    So we may write
    \begin{align*}
      \ddx a^x &= \ddx e^{x\ln(a)}\\
      &= e^{x\ln(a)} \cdot \answer[given]{\ln(a)}\\
      &= \answer[given]{a^x\cdot \ln(a)}.
    \end{align*}
  \end{explanation}
\end{theorem}

\begin{question}
  Compute:
  \[
  \ddx 7^x
  \begin{prompt}
    = \answer{7^x \ln(7)}
  \end{prompt}
  \]
\end{question}


\end{document}
