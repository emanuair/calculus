\documentclass{ximera}

\input{../preamble.tex}

\outcome{Understand the Squeeze Theorem and how it can be used to find limit values.}
\outcome{Calculate limits using the Squeeze Theorem.}

\title[Dig-In:]{The Squeeze Theorem}

\begin{document}
\begin{abstract}
The Squeeze theorem allows us to compute the limit of a difficult function by ``squeezing" it between two easy functions.
\end{abstract}
\maketitle

In mathematics, sometimes we can study complex functions by relating
them for simpler functions. The \textit{Squeeze Theorem} tells us one
situation where this is possible.

\begin{theorem}[Squeeze Theorem]\index{Squeeze Theorem}
  Suppose that
  \[
  g(x) \le f(x) \le h(x)
  \]
  for all $x$ close to $a$ but not necessarily equal to $a$. If
  \[
  \lim_{x\to a} g(x) = L = \lim_{x\to a} h(x),
  \]
  then $\lim_{x\to a} f(x) = L$.
\end{theorem}

\begin{question}
  I'm thinking of a function $f$. I know that for all $x$
  \[
  0 \le f(x) \le x^2.
  \]
  What is $\lim_{x\to 0} f(x)$?
  \begin{prompt}
  \begin{multipleChoice}
    \choice{$f(x)$}
    \choice{$f(0)$}
    \choice[correct]{$0$}
    \choice{impossible to say}
  \end{multipleChoice}
  \end{prompt}
\end{question}



\begin{example}
Consider the function
\[
f(x) =
\begin{cases}
\sqrt[5]{x}\sin\left(\frac{1}{x}\right) & \text{if $x \ne 0$,}\\
0 & \text{if $x = 0$,}
\end{cases}
\]
\begin{image}
  \includegraphics{10.png}
%% \caption[A continuous function.]{A plot of
%% \[
%% f(x)=
%% \begin{cases}
%% \sqrt[5]{x}\sin\left(\frac{1}{x}\right) & \text{if $x \ne 0$,}\\
%%  0 & \text{if $x = 0$.}
%% \end{cases}
%% \]
%% }
\end{image}
Is this function continuous at $x=0$?
\begin{explanation}
We must show that $\lim_{x\to 0} f(x) = \answer[given]{0}$. First, let's assume that $x\ge0$ and small. Since
\[
  -1 \le \sin{\left(\frac{1}{x}\right)}\le 1
  % su18(TK): sin(x) \ge 0 for small x \ge 0, but not sin(1/x).
\]

by multiplying these inequalities by $\sqrt[5]{x} \ge0$
, we obtain
\[
-\sqrt[5]{x} \le \sqrt[5]{x}\sin{\left(\frac{1}{x}\right)} \le
\sqrt[5]{x} % su18(TK): fix here
\]
which can be written as


\[
-\sqrt[5]{x} \le f(x) \le \sqrt[5]{x}. % su18(TK): here
\]
% su18(TK): comment the following out
% and, therefore as

% \[
% 0\le f(x) \le \Big|\sqrt[5]{x}\Big|
% \]

Now, let's assume that $x\le0$ and small.
Since
\[
-1\le \sin{\left(\frac{1}{x}\right)}\le 1 % su18(TK): here
\]
by multiplying these inequalities by $\sqrt[5]{x} \le0$
, we obtain
\[
\sqrt[5]{x} \le \sqrt[5]{x}\sin{\left(\frac{1}{x}\right)} \le -\sqrt[5]{x}
\]                              % su18(TK): another here
which can be written as
\[
  \sqrt[5]{x} \le f(x) \le -\sqrt[5]{x}.
\]

Therefore for all small values of x
\[
-\Big|\sqrt[5]{x}\Big| \le f(x) \le \Big|\sqrt[5]{x}\Big|.
\]

% \[
% 0\le f(x) \le \Big|\sqrt[5]{x}\Big|
% \]
Since
\[
  \lim_{x\to 0} \left( - \Big|\sqrt[5]{x}\Big|  \right)
  = \answer[given]{0} = \lim_{x\to 0}\Big|\sqrt[5]{x}\Big|
\]
 we apply the Squeeze Theorem and obtain that

$\lim_{x\to 0} f(x) = \answer[given]{0}$. Hence $f(x)$ is continuous.

Here we see how the informal definition of continuity being that you
can ``draw it'' without ``lifting your pencil'' differs from the
formal definition.

\begin{image}
  \includegraphics{11.png}
\end{image}
\end{explanation}
\end{example}


\begin{example}
Compute:
\[
\lim_{\theta\to 0} \frac{\sin(\theta)}{\theta}
\]
\begin{explanation}
To compute this limit, use the Squeeze Theorem. First note that we
only need to examine $\theta\in \left(\frac{-\pi}{2}, \frac{\pi}{2}\right)$
and for the present time, we'll assume that $\theta$ is positive. Consider
the diagrams below:

\begin{image}
\begin{tabular}{ccc}
  \includegraphics{12.png} &
  \includegraphics{13.png} &
  \includegraphics{14.png}
\end{tabular}
\end{image}


From our diagrams above we see that
\[
\text{Area of Triangle $A$} \le \text{Area of Sector} \le \text{Area of Triangle $B$}
\]
and computing these areas we find
\[
\frac{\cos(\theta)\sin(\theta)}{2} \le \frac{\theta}{2} \le \frac{\tan(\theta)}{2}.
\]
Multiplying through by $2$, and recalling that $\tan(\theta) =
\frac{\sin(\theta)}{\cos(\theta)}$ we obtain
\[
\cos(\theta)\sin(\theta) \le \theta \le \frac{\sin(\theta)}{\cos(\theta)}.
\]
Dividing through by $\sin(\theta)$ and taking the reciprocals
(reversing the inequalities), we find
\[
\cos(\theta) \le \frac{\sin(\theta)}{\theta} \le \frac{1}{\cos(\theta)}.
\]
Note, $\cos(-\theta) = \cos(\theta)$ and $\frac{\sin(-\theta)}{-\theta} =
\frac{\sin(\theta)}{\theta}$, so these inequalities hold for all $\theta\in
\left(\frac{-\pi}{2}, \frac{\pi}{2}\right)$.  Additionally, we know
\[
\lim_{\theta \to 0}\cos(\theta) = \answer[given]{1} = \lim_{\theta\to 0}\frac{1}{\cos(\theta)},
\]
and so we conclude by the Squeeze Theorem, $\lim_{\theta \to
  0}\frac{\sin(\theta)}{\theta} = \answer[given]{1}$.
\end{explanation}
\end{example}

When solving a problem with the Squeeze Theorem, one must write a sort
of mathematical poem. You have to tell your friendly reader exactly
which functions you are using to ``squeeze-out'' your limit.

\begin{example}
  Compute:
  \[
  \lim_{x\to 0} \left(\sin(x) e^{\cos\left(\frac{1}{x^3}\right)}\right)
  \]
  \begin{explanation}
    Let's graph this function to see what's going on:
    \begin{image}
      \includegraphics{15.png}
    \end{image}
    The function $\sin(x) e^{\cos\left(\frac{1}{x^3}\right)}$ has two factors:
    \begin{image}
      \includegraphics{16.png}
    \end{image}
    Hence we have that when $0< x<\pi$
    \[
   0 \le \sin(x) e^{\cos\left(\frac{1}{x^3}\right)} \le \sin(x) \answer[given]{e}
    \]
    and we see
    \[
    \lim_{x\to 0^+}0 = \answer[given]{0} = \lim_{x\to 0^+} \sin(x) \answer[given]{e}
    \]
    and so by the Squeeze theorem,
    \[
    \lim_{x\to
      0^+}\left(\sin(x)e^{\cos\left(\frac{1}{x^3}\right)}\right)=\answer[given]{0}.
    \]
    In a similar fashion, when $-\pi<x<0$,
    \[
    \sin(x) \answer[given]{e} \le \sin(x) e^{\cos\left(\frac{1}{x^3}\right)} \le 0
    \]
    and so
    \[
    \lim_{x\to 0^-}\sin(x) \answer[given]{e} =\answer[given]{0}=\lim_{x\to 0^-}0,
    \]
    and again by the Squeeze Theorem $\lim_{x\to 0^-}\left(\sin(x)
    e^{\cos\left(\frac{1}{x^3}\right)}\right)=0$. Hence we see that
    \[
    \lim_{x\to 0}\left(\sin(x)
    e^{\cos\left(\frac{1}{x^3}\right)}\right)=\answer[given]{0}.
    \]
    
         \begin{image}
      \includegraphics{17.png}
    \end{image}
  \end{explanation}
\end{example}

\end{document}
